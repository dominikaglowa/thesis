\chapter{Prologue}

\indent The composition of the world around us has been of unflagging interest to men throughout history; the earliest known records attempting to explain the composition of matter date back to antiquity. However it was only in 1911 when Rutherford formulated the theory of atomic nucleus. Ever since this discovery the research into internal structure of a nucleus accelerated. Various experimental techniques have been developed to determine the size and shape of nuclei of different elements. As a result of numerous test employing methods such as electron scattering and x-ray spectroscopy the charge distribution in a nuclei has been estimated to $\sim0.01\%$ accuracies \cite{devires}. Despite a huge success of those experimental methods in determining the proton's arrangement within a nucleus they proved to be ineffective to test the neutron distributions, as the electrons only interact with the charge in the nucleus. Strongly interacting probes such as protons, antiprotons and alpha particles have given inconsistent determinations of the nature of the neutron distributions in nuclei. The theoretical models describing these processes are still heavily subjected to large uncertainties in the description of strong interactions \cite{claire}. Currently it is not well established how the distributions of neutrons differ from protons in the nucleus.

\indent According to theoretical predictions of nuclear structure, nuclei with similar numbers of protons and neutrons have almost no difference between r.m.s. radii of charge and matter distributions. However, this situation is predicted to change with increasing mass number. It is predicted that for nuclei with the number of neutrons being much greater than the number of protons, excess neutrons form a skin around the nucleus. This neutron skin is defined as:

\begin{equation}
\Delta R = r_{n} - r_{p}
\end{equation}
where $r_{n}$ and $r_{p}$ are the r.m.s. radii of neutron and proton respectively \cite{roca}. Having $\Delta$R determined accurately from experiment for a range of nuclei will provide a stringent constraint on current nuclear structure theories. There are many different theoretical models in common use, which predict very different values for the neutron skin; and experimental measurements could provide a mean of verification of the validity of those models. Furthermore, an accurate measurement of the neutron skin will have far reaching implications in the field of nuclear physics. Currently, poorly established parameters in the equation of state for the neutron rich matter show a largely model independent linear dependence on the size of the neutron skin. Precise measurement of the neutron skin will put constraints on these parameters, which in turn will provide basis for understanding physics of neutron stars, such as the mass-radius relationship of low mass neutron stars and modified URCA cooling.

\indent Neutron skin measurements have been performed with strongly interacting probes before. The obtained data were strongly affected by many-body strong interaction effects that made the analysis and interpretation of the results ambiguous and difficult to draw binding conclusions from. The experiment involving proton scattering data fitted with different neutron skin thicknesses concluded with data being unable to determine the size or even the existence of the neutron skin \cite{pike}.

\indent The use of photons to study neutron skin potentially allows for much more accurate measurements than strongly interacting probes. The strength of photon's electromagnetic interactions is very weak and as such they are not affected by the initial state interactions (ISI) and many-body interaction effects do not complicate the interpretation of the obtained data. Furthermore, the electromagnetic interactions are far better understood than strong interactions, and therefore, the results obtained with the use of electromagnetic probes are less susceptible to systematic effects in their theoretical interpretation \cite{claire}.

\indent Coherent $\pi^{0}$ photoproduction takes place when a photon interacts with a target nucleon and as a result a $\pi^{0}$ meson is emitted, A($\gamma,\pi^{0}$)A. Even though the photons, being weakly interacting and unaffected with ISI, the produced pions are strongly interacting particles and therefore the effect of final state interactions (FSI) with the nucleus has to be accounted for. It has been shown previously that the strength of the FSI scales with pion energy and that the pion-nucleus scattering cross section is dominated by the $\Delta$(1232) resonance corresponding to the pion energy of $\sim165MeV$ . Pions with energies well away from the peak of the resonance have smaller probability of interaction with the nucleus. The study of coherent $\pi^{0}$ photoproduction provides therefore a unique way to test not only nuclear matter distribution but pion-nucleus interactions as well.

\indent The main objective of the experiment presented in this thesis is however, to use the coherent $\pi^{0}$ photoproduction as a means to study the nuclear matter distribution of three stable tin isotopes in order to determine how the neutron skin thickness evolves with the mass number. The theoretical background to the experiment and methodology are presented in the next chapter. The experimental details are presented in the subsequent chapter. Then follows the summary of the current state of knowledge and a short description of possible implications of the results presented in this thesis in the future research. The next chapters present the analysis and experimental details while the results and conclusions are discussed in the final chapters.
