\chapter{Prologue}

\indent The composition of the world around us has been of unflagging interest to humanity throughout recorded history; the earliest known records attempting to explain the composition of matter date back to antiquity. Since 1911, when Rutherford elucidated the the structure of the atom as being  a compact dense nucleus surrounded by a more diffuse electron cloud, obtainig a better understanding of the nucleus has been a major endeavour of physics. Various experimental techniques have been developed to determine the size, shape and momentum distributions of atomic nuclei, providing a wealth of data from across the nuclear chart. As a result of precise experimental methods such as electron scattering and x-ray spectroscopy the charge distribution in nuclei has been estimated with high accuracy, for example the root mean square charge radius of $^{208}$Pb is known to $\sim0.01\%$ \cite{devires}. Despite a huge success of those experimental methods in determining the distribution of protons within a nucleus, the probes employed were largely insensitive to the neutron distribution in the nucleus. This was because the electrons dominantly interact with the charge in the nucleus and are only senstive to uncharged components through the much weaker magnetic interaction. Strongly interacting probes such as protons, antiprotons and alpha particles do interact with neutrons, but their strong inetractions and the resulting short mean free path of the probe in the nucleus have resulted in inconsistent determinations of the nature of the neutron distribution. The theoretical models describing these processes are subject to large model uncertainties due to uncertainties in the description of the strong interaction in the nuclear medium \cite{claire}. Currently it is not well established how the distributions of neutrons differ from protons in the nucleus. Recent work using an electromagnetic probe and the method of coherent photoproduction of a neutral $\pi$ meson has provided new constraints for $^{208}$Pb (REF!!) with different systematics. This work will use this same technique to obtain the first assesment of how a neutron skin evolves across an isotopic chain of tin.

\indent According to theoretical predictions of nuclear structure, nuclei with similar numbers of protons and neutrons have almost no difference between r.m.s. radii of charge and matter distributions. However, this situation is predicted to change with increasing mass number. It is predicted that for nuclei with the number of neutrons being much greater than the number of protons, the excess neutrons form a small skin around the nucleus. The size of this neutron skin is usually defined as:

\begin{equation}
\Delta R = r_{n} - r_{p}
\end{equation}
where $r_{n}$ and $r_{p}$ are the r.m.s. radii of neutron and proton respectively \cite{roca}. Having $\Delta R$ determined accurately from experiment for a range of nuclei will provide a stringent constraint on current nuclear structure theories. There are many different theoretical models in common use, which predict very different values for the neutron skin; and experimental measurements could provide a means of verifying the validity of those models. Furthermore, an accurate measurement of the neutron skin will have far reaching implications in the field of nuclear physics. Currently, poorly established parameters in the equation of state for the neutron rich matter show a largely model independent linear dependence on the size of the neutron skin. Precise measurement of the neutron skin will put constraints on these parameters, which in turn will provide basis for understanding the physics of neutron stars, giving insights in properties such as the mass-radius relationship of low mass neutron stars and the feasibility of proposed cooling mechanisms such as modified URCA cooling.

\indent Neutron skin measurements have been performed with strongly interacting probes before. The obtained data were strongly affected by many-body strong interaction effects that made the analysis and interpretation of the results ambiguous and made it difficult to draw a solid conclusion. The experiments involving proton scattering data, fitted with different neutron skin thicknesses, were inconclusive on the size or even the existence of the neutron skin \cite{piek}.

\indent The use of photons to study neutron skin potentially allows for much more accurate measurements than strongly interacting probes. The strength of photon's electromagnetic is weak if compared to the strong force. For these reasons the reaction is not as much affected by the initial state interactions (ISI) and the many-body interaction effects arising from multiple scatters of the incoming probe in the nucleus do not complicate the interpretation of the obtained data. Furthermore, because the electromagnetic interactions are better understood than strong interactions, the results obtained with the use of electromagnetic probes are less sensitive to systematic effects in their theoretical interpretation \cite{claire}.

\indent Coherent $\pi^{0}$ photoproduction takes place when a photon interacts with a target nucleon and as a result a $\pi^{0}$ meson is emitted, A($\gamma,\pi^{0}$)A. Even though the photons are weakly interacting and unaffected by the ISI, the produced pions are strongly interacting particles and therefore the effect due to final state interactions (FSI) with the nucleus has to be accounted for. It has been shown previously that the strength of the FSI scales with pion energy and that the pion-nucleus scattering cross section is dominated by intermediate production of the $\Delta$(1232) resonance on the struck nucleon. Pions with energies well away from the peak of the resonance have smaller probability of interaction with the nucleus. The study of coherent $\pi^{0}$ photoproduction provides therefore a unique way to not only test nuclear matter distribution, but to assess the theoretical treatment of pion-nucleus interactions as well.

\indent The main objective of the experiment presented in this thesis is to exploit the coherent $\pi^{0}$ photoproduction as a means to study the nuclear matter distribution of three stable tin isotopes. In this way we were able to determine how the neutron skin thickness evolves increasing only the number of neutrons. The theoretical background to the experiment and methodology are presented in the next chapter. The experimental details are presented in the subsequent chapter. Then follows the summary of the current state of knowledge and a description of the implications of the results presented in this thesis for the future research. The next chapters present the analysis and experimental details while the results and conclusions are discussed in the final chapters.
