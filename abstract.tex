\chapter*{Abstract}
\noindent

\indent Heavy atomic nuclei are thought to have proton and neutron radial distributions which have different sizes. This difference is usually quantified in terms of a neutron skin (r$_{np}$), defined as the difference between the root mean square radii of the neutrons and proton radial distributions ($r_{np} = r_{n}^{2} - r_{p}^{2}$). The nature or even existence of the neutron skin is currently not well established. Different nuclear theories give different predictions for the neutron skin thickness of a typical heavy nucleus ranging from 0.05 to 0.35 fm. Accurate measurement of the properties of the neutron skin would be a powerful constraint to differentiate between models of nuclear structure and improve our knowledge of the basic Equation Of State (EOS) for neutron rich matter. Particularly, the rate at which the neutron skin thickness changes across an isotopic chain of nuclei gives a tight constraint on the EOS and is also amenable to experimental determination with small systematic error. Improving our knowledge of the EOS for neutron rich matter is a crucial step to a deeper understanding of nuclear structure and of compact astrophysical objects such as neutron stars. This thesis describes the first measurement of neutron skin  along an isotopic chain. The neutron skin is measured through the study of pions coherent photoproduction from nuclei. 

\indent This experiment was carried out in the A2 hall of the MAMI facility in Mainz, Germany in October 2012.  The incident photon beam comprised energy tagged photons in the range of $E_{\gamma}$=150-800 MeV with an intensity of ~10$^{8}$ sec$^{-1}$. Experimental data was obtained for three different tin targets, $^{116}Sn$, $^{120}Sn$ and $^{124}Sn$. The products from the resulting photoreactions were measured in the Crystal Ball detector and in the TAPS calorimeter systems, with track and particle identification information for charged particles provided by a multi wire proportional chamber (MWPC) and a particle identification detector (PID).

\indent The experiment provides the first information on the evolution of the neutron skin thickness along an isotopic chain using an electromagnetic probe. The results are compared with a range of theoretical models and previous data from strongly interacting probes. The new data provides an important experimental constraint on the basic properties of the EOS in atomic nuclei.

\indent

\vspace{10mm}
\normalsize
