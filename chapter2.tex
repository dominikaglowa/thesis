\chapter{Theoretical Background}

\indent In 1932, Heisenberg formulated an idea that proton and neutron are in fact two states of the same particle, nucleon. He proposed a new quantum number to label these states and he called it isospin ($I$) in case of a nucleon it has a value of $\frac{1}{2}$. The isospins for a proton and neutron, labelled as $I_{3}$ to convey the z-axis orientation are therefore $I_{3}=\frac{1}{2}$ and $I_{3}=-\frac{1}{2}$ respectively. Hence, the charge,$Q$, of the the nucleon can be written as:

\begin{equation}
\frac{Q}{e}=\frac{1}{2}+I_{3}
\end{equation}
where $e$ is the charge of an electron.

\indent In 1935, Yukawa, proposed that in order for the nucleons to be held together in a nucleus, some kind of "strong" force must exist and that the mediator particle for this force should be a spin-0 meson with a mass of $\sim150MeV$ \cite{dudek}. The existence of such particle has been proven 12 years later in 1947 when the charged pions were discovered by the collaboration of C. Powell, C. Lattes and G. Occhialini \cite{martin}.

\indent Pions are the lightest mesons with mass of $\sim135MeV$ and they act as mediators of the long-range part of the strong nuclear force. They are zero spin particles composed of the first generation of quarks and they can be found in three states, neutral ($\pi^{0}$) and charged ($\pi^{+}$ and $\pi^{-}$).

\indent Charged pions decay via weak interactions into muon and neutrino, the branching ratio of this decay mode is $\sim99.9\%$ with a mean lifetime of $2.6 \times 10^{-8}$s.
\begin{equation}
\pi^{+} \rightarrow \mu^{+} + \nu_{\mu}
\end{equation}
\begin{equation}
\pi^{-} \rightarrow \mu^{-} + \bar{\nu_{\mu}}
\end{equation}
Another decay mode is into electron or positron and electron anti-neutrino or neutrino respectively, however, the probability of this decay channel is very low $\sim0.001\%$.

\indent Neutral pions decay via electromagnetic interactions with a mean lifetime of $8.4 \times 10^{-17}$s. The dominant decay mode (branching ratio of $\sim99\%$) is into two photons:

\begin{equation}
\pi^{0} \rightarrow \gamma + \gamma
\end{equation}
In this case, the second most probable decay mode ($\sim1\%$) is into a photon and an electron-positron pair \cite{amsler}.

\section{Coherent pion photoproduction}

\indent Pion photoproduction occurs when photon interacts with a nucleon and causes it to emit pions. There are four possible channels for such reaction to take place:

\begin{equation}
\gamma + p \rightarrow p + \pi^{0}
\end{equation}
\begin{equation}
\gamma + p \rightarrow n + \pi^{+}
\end{equation}
\begin{equation}
\gamma + n \rightarrow p + \pi^{-}
\end{equation}
\begin{equation}
\gamma + n \rightarrow n + \pi^{0}
\end{equation}

The process can only happen when the nucleon is bound within a nucleus. And a coherent process occurs only if the target nucleus is left in its ground state, $A_{gs}(\gamma,\pi^{0})A_{gs}$. If the initial and final states differ the process is incoherent. Because of the charge conservation, reactions involving charged pions leave the target nucleus in a different state than the original, therefore; the only coherent production possible is the one featuring neutral pions \cite{claire}.

\indent The $\pi^{0}$ production process occurs with similar probability on both protons and neutrons. In the case of coherent reaction the amplitudes from all nucleons add coherently and the resulting cross section is directly proportional to the square of mass number, A, and the square of the matter form factor. In the Plane Wave Impulse Approximation (PWIA) the cross section is expressed as:

\begin{equation}
\frac{d\sigma}{d\Omega}=A^{2} \frac{q}{k_{\gamma}} P^{2}_{3} |F_{m}(q)|^{2} sin^{2}(\theta_{\pi})
\end{equation}
where q is the momentum transfer to the nucleus, $F_{m}(q)$ is the matter form factor of the nucleus, $P_{3}$ is the contributing pion photoproduction amplitude \cite{prop}. The matter form factor is a Fourier transform of the matter density distribution and because of that a diffraction pattern can be observed in the differential cross section \cite{claire}.

\indent The use of photons  to study neutron skin potentially allows for much more accurate measurements than strongly interacting probes. The strength of photon's electromagnetic interactions is very weak and as such they are not affected by the initial state interactions (ISI) and many-body interaction effects do not complicate the interpretation of the obtained data. Furthermore, the electromagnetic interactions are far better understood than strong interactions, and therefore, the results obtained with the use of electromagnetic probes are less susceptible to systematic effects in their theoretical interpretation
\cite{claire}.

\indent Despite the photon being weakly interacting and therefore not affecting the pion photoproduction experiments with ISI the produced pions are strongly interacting particles and therefore the effect of final state interactions (FSI) with the nucleus as to be accounted for. The real part of the interaction is responsible for an angular shift in the ($\gamma,\pi^{0}$) angular distribution and the imaginary part, taking the absorption processes into account, explains the reduction in flux \cite{Proposal}. It has been shown previously that the strength of the FSI scales with pion energy and that the pion-nucleus scattering cross section is dominated by the $\Delta(1232)$ resonance corresponding to the pion energy of $\sim165MeV$. Pions with energies well away from the peak of the resonance have smaller probability of interaction with the nucleus.

\indent Although the FSI complicates the analysis of the matter distribution in the nucleus they also provide a very effective way to study pion-nucleus interaction across the whole volume of the nucleus. All the available information about those interactions comes from charged pions scattering experiments but due to very short lifetimes pions with energies lower than 30MeV could not have been employed in those measurements. However, coherent $\pi^{0}$ production is not constrained with the limitations of charged pions scattering experiments. It offers an opportunity to investigate the pion-nucleus interaction for pion energies nearing 0MeV evaluated over the entire volume of the nucleus \cite{claire}.

\indent The study of coherent $\pi^{0}$ photoproduction provides therefore a unique way to test not only nuclear matter distribution but pion-nucleus interactions as well. The main objective of the experiment presented in this report is however to use the coherent $\pi^{0}$ photoproduction as a means to study the nuclear matter distribution of tin isotopes in order to determine how the neutron skin thickness depends on mass number. The predictions of neutron skin thicknesses of tin isotopes are presented in Fig. \ref{tiniso} below.

\indent Across an isotopic chain from 112 to 124 we expect a change in the neutron skin of 0.05fm which should be easily detectable in the measurement.

\begin{figure}[H]
\begin{center}
\includegraphics[scale=0.6]{pictures/png/tiniso.png}
\caption{The predictions of the neutron skin thicknesses for tin isotopes from the IU-FSU and SkIU-FSU models after the optimisation with those from different experimental methods \cite{fattoyev}.}
\label{tiniso}
\end{center}
\end{figure}

\subsection{Reaction kinematics}

\indent Schematics of the kinematics of a pion photoproduction reaction are shown in a figure below (Fig. \ref{photorea}). The interacting particles, photon ($\gamma$) and a nucleon (N) have initial four-momenta k and $p_{i}$ respectively. The four-momentum is a combination of particle's energy and its three-momentum: P=(p,E).

\begin{figure}[H]
\begin{center}
%\includegraphics[scale=0.6]{photorea.png}
\caption{Simple diagram of a photoproduction reaction.}
\label{photorea}
\end{center}
\end{figure}

\indent The Feynman diagrams shown below, illustrate that this reaction can proceed via three possible mechanisms (Fig. \ref{mandelstam}). First diagram, s-channel describes a process where photon and nucleon combine and an intermediate particle (resonance) is formed. In the case of t-channel, one of the interacting particles emits an intermediate particle which is subsequently absorbed by the other interacting particle. U-channel describes the same situation as t-channel with the exception of the final state particles being interchanged.

\begin{figure}[H]
\begin{center}
%\includegraphics[scale=0.6]{mandelstam.png}
\caption{Feynman diagrams of the s-channel, t-channel and u-channel \cite{jo}.}
\label{mandelstam}
\end{center}
\end{figure}

\indent The kinematics of such production reaction can be easily described with the use of Lorentz-invariant Mandelstam variables s, t and u. The mechanisms are commonly referred to as: s-channel, t-channel and u-channel (Fig. \ref{mandelstam}). They are defined in terms of four-momenta as \cite{walker}:

\begin{equation}
s=(k+p_{i})^{2}=(q+p_{f})^{2}
\end{equation}
\begin{equation}
t=(p_{i}-p_{f})^{2}=(k-q)^{2}
\end{equation}
\begin{equation}
u=(p_{i}-p_{f})^{2}=(k-p_{f})^{2}
\end{equation}
s is the square of the energy of the reaction, t is the square of the momentum transfer, while the sum of the squares of the masses of particles is defined as a linear combination of those three variables.

\begin{equation}
s+t+u=\sum m^{2}_{i}
\end{equation}

\indent Only two of the Mandelstam variables are necessary to fully describe the reaction between photon and nucleon. In the relativistic approximation, $mc^{2}<<E$, these variables can be represented as:

\begin{equation}
s=4p^{2}
\end{equation}
\begin{equation}
t=2p^{2}(1-cos\theta)
\end{equation}
\begin{equation}
u=2p^{2}(1+cos\theta)
\end{equation}
where $\theta$ is the pion scattering angle in the centre of mass frame of reference. If $s$ is fixed, t is a linear function of $cos\theta$, and therefore, the scattering functions can be represented completely in terms of $s$ and $cos\theta$ \cite{bertulani}.

\subsection{Reaction cross sections}

\indent For a pion, or any meson, created in a photoproduction process, its angular distribution can be represented by a differential cross section:

\begin{equation}
\frac{d\sigma}{d\Omega}=|A(s,cos\theta)|^{2}
\end{equation}

\indent The probability of the transition from a given initial state $<i|$ to another final state $|f>$ can be represented by a scattering matrix S in the Bjorken Drell notation\cite{bjorken}:

\begin{equation}
S_{fi}=\delta_{}fi-\frac{i}{(2\pi)^{2}}\delta^{4}(q-k+p_{f}-p_{i})\sqrt{\frac{m_{N}^{2}}{4E_{\gamma}E_{\pi}E_{i}E_{f}}}<i|T|f>
\end{equation}
where, $T$ is the transmission matrix relating initial and final states, $m_{N}$ is the mass of a nucleon, and q, k, $p_{i}$, $p_{f}$ are four-momenta of the involved particles. The transition matrix describes the amplitude of the photoproduction process, and can be written as:

\begin{equation}
T=\epsilon_{\mu}J_{\mu}
\end{equation}
where, $\epsilon_{\mu}$ is the photon polarisation vector and $J_{\mu}$ is the electromagnetic current of a nucleon. Then the differential cross section can be defined ac:

\begin{equation}
\frac{d\sigma}{d\Omega}=\frac{q}{k}(\frac{m_{N}}{4\pi W})^{2}\sum |T|
\end{equation}
where, W is the invariant mass of the system.

\indent The electromagnetic current of a nucleon, $J$, as proposed by Chew, Goldberg, Low, Nambu (CGLN), can be written in terms of nucleon spin matrices and meson unit vectors \^q and \^k \cite{chew}:

\begin{equation}
J=%\frac{q}{k}\frac{4\pi W}{m_{n}}(i\sigma F_{1}+(\sigma \dot \^k)(\sigma \cross \^q)F_{2}+i\~ k(\sigma \dot \^q)F_{3}+i\~ k(\sigma \dot \^k)F_{4})
\end{equation}
where,
\begin{equation}
^{\sim}\sigma=%\sigma-(\sigma \dot \^q)\^q
\end{equation}
\begin{equation}
^{\sim} k=%\^k-(\^k \dot \^q)\^q
\end{equation}
and $F_{1}$, $F_{2}$, $F_{3}$, $F_{4}$, known as CGLN amplitudes, are the structure functions of energy and scattering angle, which can be written in terms of electric and magnetic multipoles and angular momentum through a partial wave expansion:

\begin{equation}
F_{1}(\theta)=
%\sum_{l=0}^{\infinity}(lM_{l+}+E_{l+})P'_{l-1}(cos\theta)+((l+1)M_{l-}_E_{l-})P'_{l-1}(cos\theta)
\end{equation}
\begin{equation}
F_{2}(\theta)=
%\sum_{l=0}^{\infinity}((l+1)M_{l+}+lM_{l-})P'_{l}(cos\theta)
\end{equation}
\begin{equation}
F_{3}(\theta)=
%\sum_{l=0}^{\infinity}(E_{l+}-M_{l+})P"_{l+1}(cos\theta)+((E_{l-}+M_{l-})P"_{l-1}(cos\theta)
\end{equation}
\begin{equation}
F_{4}(\theta)=
%\sum_{l=0}^{\infinity}(M_{l+}-E_{l+}-M_{l}-E_{l-})P"_{l}(cos\theta)
\end{equation}
where, $P'_{l}$ and $P"_{l}$ are derivatives of a Legendre polynomials, $l$ is the relative orbital momentum of a meson, $E_{+/-}$ and $M_{+/-}$ are electric and magnetic transition respectively, and the + or - determine whether the spin of the baryon should be added or subtracted.

\indent The total angular momentum of a nucleon in $\frac{1}{2}$, and the total angular momentum of an incident photon is $L_{\gamma}$. The resulting spin of a resonant state, in order to satisfy the selection rule, has to obey the following relation \cite{krusche}:

\begin{equation}
|L_{\gamma}-\frac{1}{2}|<J_{N*}<|L_{\gamma}+\frac{1}{2}|
\end{equation}
and the parity is given as:

\begin{equation}
\pi_{N*}=\pi_{N}\pi_{\gamma}
\end{equation}
where, $\pi_{N}$, parity of a nucleon, i.e. equal to 1, and $\pi_{\gamma}$, the parity of the photon is equal to:

\begin{equation}
\pi_{\gamma}=(-1)^{L_{\gamma}}
\end{equation}
\begin{equation}
\pi_{\gamma}=(-1)^{L_{\gamma}+1}
\end{equation}
respectively for an electric and magnetic multipoles.

\indent The selection rules for the angular momentum and parity of the resonant state with respect to the outgoing meson are given as:

\begin{equation}
|L_{\pi}-\frac{1}{2}|<J_{N*}<|L_{\pi}+\frac{1}{2}|
\end{equation}
\begin{equation}
\pi_{N*}=\pi_{N}\pi_{\pi}(-1)^{L_{\pi}}=(-1)^{L_{\pi}+1}
\end{equation}
where $\pi_{\pi}$ is -1.

\indent Combining the above equations sets a limit on the spin and parity of the resonance:

\begin{equation}
|L_{\gamma}+/-\frac{1}{2}|<J_{N*}<|L_{\gamma}+/-\frac{1}{2}|
\end{equation}
\begin{equation}
\pi_{N*}=\pi_{N\gamma}=(-1)^{L_{\pi}+1}
\end{equation}

\indent When a cross section is dominated by a single resonance, its quantum numbers are reflected in the angular distribution because the electric multipole $L={L_{\pi}+/-1}$ and magnetic multipole $L={L_{\pi}}$, the spin and parity are related to the angular distribution of mesons for the dependence of the Legendre polynomials in the CGLN amplitudes. Most photoproduction mechanism, however, have more than one multipole contributing to the resonance and in order to tell all the different contributions apart, several channels must be measured.

\indent Isospin, not conserved in the electromagnetic interactions, is conserved however, in the hadronic interactions. The isospin of the initial state, determined from the isospin of the nucleon, is $I_{i}=\frac{1}{2}$ while the contributions from the pion and nucleon's isospins determine the one of the final state, $I_{f}$, which can therefore take values of $\frac{1}{2}$ or $\frac{1}{2}$. The value is determined by the photon which behaves as a linear combination of isoscalar ($I_{s}$) which conserves the isospin and isovector ($I_{v}$) which can change the isospin by one, components. The whole system can be then described by the three isospin amplitudes \cite{nagl}:

\begin{equation}
A^{0}=<\frac{1}{2},I_{3}|I_{s}|\frac{1}{2},I_{3}>
\end{equation}
for the isoscalar electromagnetic current
\begin{equation}
A^{1}=<\frac{1}{2},I_{3}|I_{v}|\frac{1}{2},I_{3}>
\end{equation}
\begin{equation}
A^{3}=<\frac{3}{2},I_{3}|I_{v}|\frac{1}{2},I_{3}>
\end{equation}
while for the isovector electromagnetic current, the amplitudes are written as \cite{davidson}:

\begin{equation}
A^{+}=\frac{A^{1}+2A^{3}}{3}
\end{equation}
\begin{equation}
A^{-}=\frac{A^{1}-A^{3}}{3}
\end{equation}

\indent In terms of isospin amplitudes, the physical amplitudes for pion photo production are written as:

\begin{equation}
A(p\gamma \rightarrow n\pi^{0})=<p\pi^{0}|I|p\gamma>=(A^{0}+A^{+})
\end{equation}
\begin{equation}
A(n\gamma \rightarrow n\pi^{0})=<n\pi^{0}|I|n\gamma>=(A^{+}+A^{0})
\end{equation}
\begin{equation}
A(p\gamma \rightarrow n\pi^{+})=<n\pi^{+}|I|p\gamma>=\sqrt{2}(A^{0}+A^{-})
\end{equation}
\begin{equation}
A(n\gamma \rightarrow p\pi^{-})=<p\pi^{-}|I|n\gamma>=\sqrt{2}(A^{0}-A^{-})
\end{equation}
The above set of equations allows for the separation of the amplitudes for individual photoproduction reactions provided that the measurements on both proton and neutron targets are carried out.

\subsection{Partial Wave Analyses}

\indent In order to extract information on the masses, widths and amplitudes from the experimental data, different reaction models have been developed and used. The most common approach involves separation of background and resonant terms of a transition matrix.

\indent Considering a reaction of the form $A\rightarrow B\rightarrow C$, where $A$ is the initial state of the nucleon-photon system,$B$ is the intermediate resonant state and $C$ is the final state of the nucleon-meson system, the photoproduction process can be described with a below Hamiltonian:

\begin{equation}
H=H_{0}+V_{bg}+V_{R}(E)
\end{equation}
where, $H_{0}$ is a free Hamiltonian expressing the total kinetic energy of the interacting particles, $V_{bg}$ is the potential due to the background created by the non-resonant contributions to the reaction, and $V_{R}$ is potential due to resonant term.

\indent The transmission matrix for the process is given as:

\begin{equation}
T_{AC}=V_{AC}+\sum_{B}V_{AB}g_{B}(E)T_{BC}(E)
\end{equation}
where, $g_{B}$ is the propagator of the channel B of the reaction, and $\sum_{B}$ sums all the possible channels of the reaction $A\rightarrow C$ via $B$. Alternatively, the transmission matrix can be split into the background and resonant terms what allows for independent calculations of for the background and resonant contributions.

\begin{equation}
T_{AC}=t_{bg}^{AC}+t_{R}^{AC}(E)
\end{equation}

\indent Partial wave analyses (PWA) are methods that allow for decomposition of the background and resonant terms of the transmission matrix into a number of partial waves of defined multipoles and momentum. Generally, the resonant terms are modelled with a Breit-Wigner form while the background is described with Born terms and vector-meson contributions. The extractions of the parameters from the analysis is done with a two-stage procedure of fitting the experimental data. The most commonly used for the pion photoproduction PWAs are MAID, developed at the University of Mainz \cite{maid}, and SAID written by the CNS Data Analysis Centre at George Washington University \cite{said}.

\indent MAID is a unitary isobar model which describes the transmission matrix through a single $\pi N$ channel \cite {drechsel}:

\begin{equation}
T_{\gamma\pi}=V_{\gamma\pi}(E)+V_{\gamma\pi}(E)g_{B}(0)T_{\pi N}(E)
\end{equation}
where, $V_{\gamma\pi}$ is the transition potential of the $\gamma N\rightarrow \pi N$ reaction, $T_{\pi N}$ and $g_{0}$ are the scattering matrix and the free propagator of the $\pi N$ interaction respectively.

\indent The scattering matrix and the transition potential can be broken down to its constituents background and resonant terms and those can be expanded as partial waves. The resonances included in MAID, classified as 4* by the Particle Data Group (PDA) \cite{pda}, are all only up to $2GeV$.

\indent In the methods developed for SAID, there are no assumptions about resonances and channels included in the analysis framework. The transmission matrix, defined for the following three channels: $\gamma N$, $\pi N$ and $\pi \Delta$ (covering all open channels), can be written as \cite{arndt}:

\begin{equation}
T_{\gamma\pi}=A_{1}(1+iT_{\pi N})+A_{R}T_{\pi N,\pi N}
\end{equation}
where $A_{R}$ parametrises the multipole amplitudes of the resonant terms, and $A_{1}$ parametrises background.

\begin{equation}
A_{R}=\frac{m_{\pi}}{q}[\frac{k}{q}]^{l}\sum_{n=0}^{N}p_{n}[\frac{E_{\pi}}{m_{\pi}}]
\end{equation}
\begin{equation}
E_{\pi}=\frac{s-(m_{\pi}+M)^{2}}{2M}=E_{\gamma}-m_{\pi}(1+\frac{m_{\pi}}{2M})
\end{equation}
where $E_{\pi}$ is the pion kinetic energy in the lab frame for the $\pi N\rightarrow \gamma N$, s is the square of the centre of mass energy, $M$ is the mass of the nucleon and $E_{\gamma}$ is the photon lab energy for $\gamma N \rightarrow \pi N$, and $p_{n}$ is a free parameter determined in the fit to the experimental data.

\begin{equation}
A_{1}=A_{B}+A_{Q}
\end{equation}
where, $A_{B}$ is a partial wave of a pseudoscalar Born amplitude, and $A_{Q}$ is a Legendre function.

\indent Detailed explanation of MAID and SAID can be found in \cite{maid} and \cite{said, arndt}.

\section{Nuclear Matter Distribution}

\indent ?

\subsection{The nuclear Equation of State}

\indent By definition, the equation of state (EOS) is a function of density ($\rho$) and isospin asymmetry ($\alpha$) and it's expressed as the energy (E) per nucleon in the infinite nuclear matter, $\frac{E}{A} (\rho,\alpha)$ where $\alpha$ is written as:

\begin{equation}
\alpha=\frac{N-Z}{A}
\end{equation}
where $N$ is the number of neutrons,$Z$ is the number of protons and $A$ is the atomic mass number.



\section{Previous Measurements}

\indent Neutron skin measurements have been performed with strongly interacting probes before. The obtained data were strongly affected by many-body strong interaction effects that made the analysis and interpretation of the results ambiguous and difficult to draw binding conclusions from. The experiment involving proton scattering data fitted with different neutron skin thicknesses concluded with data being unable to determine the size or even the existence of the neutron skin \cite{piek}.

\indent The measurement of parity violation in electron scattering provides an independent probe of neutron densities because the weak charge of the neutron is much larger than that of proton. The interpretation of such results is model-independent and unaffected by the uncertainties of strong interactions. The polarised electrons have been used as a probe of neutron distribution in the $^{208}$Pb Radius Experiment (PREX). It made use of parity violation to accurately determine the neutron radius of the lead nucleus. The first measurement of the parity-violating asymmetry, $A_{PV}$ in the elastic scattering of polarised electrons from $^{208}$Pb reports the thickness of neutron skin of $\Delta R = 0.33_{-0.18}^{+0.16}$fm, and therefore provides the first electroweak observation of the neutron skin. \cite{prex}.

\indent The experiment investigating nuclear periphery at Low Energy Antiproton Ring (LEAR) at CERN allowed for an estimate of the relation between the symmetry parameter and neutron skin. 26 isotopes with mass numbers in a range of 40-238 have been studied. Antiprotons were chosen as a probe for the experiment since the antiproton-nucleus interactions are very strong and even a small overlap between their wave functions leads to the annihilation of the antiproton with one of the peripheral nucleons resulting in a final state nucleus with either proton or neutron number lower by one unit compared to the initial state. If both products are radioactive nuclear spectroscopy can be employed to determine their relative yields which are directly related to the proton and neutron densities at the annihilation site. The experimental results are presented in (Fig. \ref{antiproton}); analysis of the data indicated that neutrons are distributed in a form of a halo rather than a skin \cite{trzcina}. However, as the antiprotons are absorbed only in the surface of the nucleus the systematics in such measurements are somewhat controversial.

\begin{figure}[H]
\begin{center}
\includegraphics[scale=0.6]{pictures/png/antiproton.png}
\caption{Difference between the r.m.s. radii of the neutron and proton distributions ($\Delta r_{np}$) against the symmetry parameter $\delta$. Picture taken from the reference \cite{trzcina}.}
\label{antiproton}
\end{center}
\end{figure}



%now here you can refer to the first section like, see section \ref{sec1}......

